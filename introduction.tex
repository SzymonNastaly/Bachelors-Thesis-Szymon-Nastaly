\chapter{Introduction} \label{chap:introduction}

The internet has become a central point of personal data collection~\cite{kretschmer2021cookie}.
Users are spending long periods of time on the internet (the average European more then two hours per day~\cite{kretschmer2021cookie}) and are being widely tracked while doing so (on average 17 trackers per website~\cite{kretschmer2021cookie}).
This has led to the introduction of regulations such as the General Data Protection Regulation (\emph{GDPR}). 
Enacted in May 2018, it provides a comprehensive privacy legislation across the European Union (\emph{EU}).
Expanding on previous regulations (such as the ePrivacy Directive), it has defined additional restrictions for data collection, responsibilities for website owners and rights for users~\cite{kretschmer2021cookie}.

Websites need to inform users about data collection and its purposes.
Additionally, a legal basis such as user consent is needed for non-functional cookies which e.g., may be used for user tracking.
Such regulation is highly relevant as more then 90\% of websites use cookies capable of user identification.
In the context of browser cookies, both criteria are commonly covered via cookie notices~\cite{bouhoula2023automated}.
In spite of the potential fines~\cite{sanchez_rola2019can}, violations have been very common with 65\% of websites ignoring user rejection of cookies~\cite{bouhoula2023automated}.
An overview of large-scale studies regarding the adherence to GDPR rules was compiled by Bouhoula et al.~\cite{bouhoula2023automated}.

It can be challenging to verify compliance of specific websites for web page operators, users and regulators with these regulations.
Cookie notices often have different designs and implementation details.
This complicates automatic verification whether a notice defines the cookie purposes and offers possibilities to configure or reject non-necessary cookies.
Large-scale studies have therefore relied on heuristics, manual annotation and machine learning models~\cite{kretschmer2021cookie, bouhoula2023automated}.
Similarly, recognizing what cookies are used for is not trivial~\cite{sanchez_rola2019can, bollinger2022automating}.

Tools designed for users to audit specific websites are either fully manual and labor intensive~\cite{gorin2024edpb}, can only automatically analyze cookie notices by specific Consent Management Platforms (\emph{CMP}), require IAB TCF compliance~\cite{matte2020cookiebannersrespectchoice}, or only provide rudimentary analysis~\cite{cookie_information2019cookie}.

\section{Our Work}
We aim to provide a tool that is both applicable on a wide variety of websites
We aim to provide a general tool that can be used to audit the cookie usage of websites for compliance with GDPR regulation.
We address the limitations of previous tools which were restricted to certain Consent Management Platforms (\emph{CMP}), required IAB TCF compliance~\cite{matte2020cookiebannersrespectchoice}