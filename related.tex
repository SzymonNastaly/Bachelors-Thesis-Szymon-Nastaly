\chapter{Related Work}

Several studies have investigated the violations of privacy-regulations by websites.
The impact of the 2018 GDPR has been a particular focus of research.
The meta-study by Kretschmer et al.~\cite{kretschmer2021cookie} provides an overview into the observed, quantitative effects on cookie compliance and privacy policies.

The following two studies are especially relevant to our work.

Bollinger et al.~\cite{bollinger2022automating} study cookie consent practices and GDPR compliance on websites using specific Consent Management Platforms (CMPs). 
They observe that 69.7\% of websites set non-essential cookies before obtaining user consent. 
For their analysis, they develop a gradient boosting model to classify cookies into Essential, Functional, Analytics, and Advertising categories.

Building on this, Bouhoula et al.~\cite{bouhoula2023automated} create a fully automated, CMP-independent approach for large-scale analysis of GDPR cookie compliance. 
In addition to using the cookie classification model by Bollinger et al., they developed NLP models to analyze cookie notice text and interactive elements. 
In their comprehensive set of 97k popular websites targeting EU users, they find 65.4\% of pages still collecting user data after an explicit rejection.


\section{Inspection of Individual Web Pages}
So far researchers have mostly developed large-scale approaches to investigate general trends in GDPR compliance. 
An overview of many of those studies is presented by Bouhoula et al.~\cite{bouhoula2023automated}.
At the same time there is a need by website operators but also regulators for auditing tools that verify compliance of websites.
Next, we will give an overview of the existing tools while also highlighting their particular limitations.

\subsection{European Data Protection Board Website Auditing Tool}
\emph{EDBP WAT} is an auditing tool developed by Jérôme Gorin for the European Data Protection Board~\cite{gorin2024edpb}.
It uses a custom Chromium interface to analyze how sites collect and store browser data, as well as inspect network traffic. 
%Users can create analysis sessions with multiple scenarios and then manually interact with the website.
Users can record different scenarios, e.g., the rejection of cookies and subsequent manual interaction with the website.
EDBP then categorizes the collected cookies, local storage, web beacons\footnote{
A web beacon (or tracking pixel) is a method for tracking users. 
Visitors can be monitored by including an image (often invisible) from an external server inside the website.~\cite{smith1999web}.
} and HTTP requests by matching against tracker databases.
The tool generates customizable reports to export the findings.

\subsection{Cookie Glasses}
Matte et al.~\cite{matte2020cookiebannersrespectchoice} have investigated the compliance of cookie banners implementing IAB Europe’s Transparency and Consent Framework (TCF) and found that they did not always respect the users choice.
As part of the research the browser extension \emph{Cookie Glasses} was developed.
It aimed to help website developers verify their compliance.
It was restricted to TCF cookie notices and in particular TCF version 1 (v2.2 is most recent).
As of July 2024, the development of Cookie Glasses is abandoned.

\subsection{Privacy Pioneer}
Zimmeck et al.~\cite{zimmeck2024pioneer} designed \emph{Privacy Pioneer}, a Firefox-only browser extension for real-time detection of GDPR relevant data collection.
The extension monitors web traffic by analyzing HTTP messages using the \verb|webRequest| API.
It searches the traffic for location information, monetization practices, tracking methods and personal data. 
For this, it uses a combination of regular expression matching, URL list comparisons and a machine learning model trained to detect location data.
In contrast to CookieAudit, Privacy Pioneer does not analyze cookies or cookie notices.

