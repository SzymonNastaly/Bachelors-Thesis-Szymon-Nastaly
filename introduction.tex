\chapter{Introduction} \label{chap:introduction}

The General Data Protection Regulation (\emph{GDPR}), enacted in May 2018, introduced comprehensive privacy legislation across the European Union, impacting how websites handle personal data and user consent.
Websites need to inform users about data collection and its purposes.
Additionally, a legal basis such as user consent is needed for non-functional cookies which e.g., may be used for user tracking.
Such regulation is highly relevant as more then 90\% of websites use cookies capable of user identification.
In the context of browser cookies, both criteria are commonly covered via cookie notices~\cite{bouhoula2023automated}.
In spite of the potential fines~\cite{sanchez_rola2019can}, violations have been very common with 65\% of websites ignoring user rejection of cookies~\cite{bouhoula2023automated}.
An overview of studies regarding the implementation of GDPR rules was created by Bouhoula et al.~\cite{bouhoula2023automated}.

It can be challenging to verify compliance of specific websites for web page operators, users and regulators with these regulations.
Cookie notices often have different designs and implementation details.
This complicates automatic verification whether a notice defines the cookies purposes and offers possibilities to configure or reject non-functional cookies.
Large-scale studies have therefore relied on heuristics, manual annotation and machine learning models~\cite{kretschmer2021cookie, bouhoula2023automated}.
Similarly, recognizing what cookies are used for is not trivial and can be done with entropy calculations~\cite{sanchez_rola2019can} or machine learning models~\cite{bollinger2022automating}.

\section{Our Work}
We aim to provide a general tool that can be used to audit the cookie usage of websites for compliance with GDPR regulation.
We address the limitations of previous tools which were restricted to certain Consent Management Platforms (\emph{CMP}), required IAB TCF compliance~\cite{matte2020cookiebannersrespectchoice}