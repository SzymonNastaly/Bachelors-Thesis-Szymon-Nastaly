\chapter{Related Work}

\textcolor{red}{wip}

\section{Evaluation of broad GDPR Compliance}
Since the introduction of the GDPR (see \cref{sec:legal}) there has been research that examined the changes and compliance of websites.
\textcolor{red}{We have to name some concrete research that spans: non-interacting testing, manual testing, CMP focused testing, maybe also country focused testing.}. 
Those papers have aimed to show a summarized pictured of adherence across many websites.

\section{Inspection of Individual Web Pages}
There have also been tools aimed at users for evaluating the compliance of individual websites. 
They have also lacked in generality (because they focused on certain CMPs), ...

\subsection{European Data Protection Board Website Auditing Tool}
\emph{EDBP WAT} is an auditing tool developed by Jérôme Gorin for the European Data Protection Board~\cite{gorin2024edpb}.
It uses a custom Chromium interface to analyze how sites collect and store browser data, as well as inspect network traffic. 
%Users can create analysis sessions with multiple scenarios and then manually interact with the website.
Users can record different scenarios, e.g., the rejection of cookies and subsequent manual interaction with the website.
EDBP then categorizes the collected cookies, local storage, web beacons\footnote{
A web beacon (or tracking pixel) is a method for tracking users. 
Visitors can be monitored by including an image (often invisible) from an external server inside the website.~\cite{smith1999web}.
} and HTTP requests by matching against tracker databases.
The tool generates customizable reports to export the findings.

\subsection{Cookie Glasses}
Matte et al.~\cite{matte2020cookiebannersrespectchoice} have investigated the compliance of cookie banners implementing IAB Europe’s Transparency and Consent Framework (TCF) and found that they did not always respect the users choice.
As part of the research the browser extension \emph{Cookie Glasses} was developed.
It aimed to help website developers verify their compliance.
It was restricted to TCF cookie notices and in particular TCF version 1 (v2.2 is most recent).
As of July 2024, the development of Cookie Glasses is abandoned.

\subsection{Privacy Pioneer}
\emph{Privacy Pioneer} (by Zimmeck et al.~\cite{zimmeck2024pioneer}) is a Firefox-only browser extension for real-time detection of GDPR relevant data collection.
The extension monitors web traffic by analyzing HTTP messages using the \verb|webRequest| API.
It searches the traffic for location information, monetization practices, tracking methods and personal data. 
For this, it uses a combination of regular expression matching, URL list comparisons and a machine learning model trained to detect location data.
In contrast to CookieAudit, Privacy Pioneer does not analyze cookies or cookie notices.

