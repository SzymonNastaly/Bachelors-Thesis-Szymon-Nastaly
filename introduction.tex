\chapter{Introduction} \label{chap:introduction}

The internet has become a central point of personal data collection~\cite{kretschmer2021cookie}.
Users are spending long periods of time on the internet (the average European more then two hours per day~\cite{kretschmer2021cookie}) and are being widely tracked while doing so (on average 17 trackers per website~\cite{kretschmer2021cookie}).
This has led to the introduction of regulations such as the General Data Protection Regulation (\emph{GDPR}). 
Enacted in May 2018, it provides a comprehensive privacy legislation across the European Union (\emph{EU}).
Expanding on previous regulations (such as the ePrivacy Directive), it has defined additional restrictions for data collection, responsibilities for website owners and rights for users~\cite{kretschmer2021cookie}.

Websites need to inform users about data collection and its purposes.
Additionally, a legal basis such as user consent is needed for non-functional cookies which e.g., may be used for user tracking.
Such regulation is highly relevant as more then 90\% of websites use cookies capable of user identification.
In the context of browser cookies, both criteria are commonly covered via cookie notices~\cite{bouhoula2023automated}.
In spite of the potential fines~\cite{sanchez_rola2019can}, violations have been very common with 65\% of websites ignoring user rejection of cookies~\cite{bouhoula2023automated}.
An overview of large-scale studies regarding the adherence to GDPR rules was compiled by Bouhoula et al.~\cite{bouhoula2023automated}.

It can be challenging to verify compliance of specific websites for web page operators, users and regulators with these regulations.
Cookie notices often have different designs and implementation details.
This complicates automatic verification whether a notice defines the cookie purposes and offers possibilities to configure or reject non-necessary cookies.
Possible approaches include heuristics, manual annotation and machine learning models~\cite{kretschmer2021cookie, bouhoula2023automated}.
Similarly, recognizing what cookies are used for is not trivial~\cite{sanchez_rola2019can, bollinger2022automating}.
Those issues become apparent in previous large-scale studies:
Nouwens et al.~\cite{nouwens2020dark} restrict their study to websites implementing five popular Consent Management Platforms (\emph{CMP}).
After collecting data from 680 websites from the United Kingdom they find that only 11.80\% of cookie notices met basic GDPR requirements (namely, no pre-ticked optional boxes, simple way to reject and explicit consent).
Matte et al.~\cite{matte2020cookiebannersrespectchoice} focus on cookie notices adhering to the IAB TCF specification.
They describe violations such as "Implicit consent prior to interaction" in 10\% of websites.

Similarly, tools designed for users to audit specific websites are either fully manual and labor intensive~\cite{gorin2024edpb}, can only automatically analyze cookie notices by specific CMPs, require IAB TCF compliance~\cite{matte2020cookiebannersrespectchoice}, or only provide rudimentary analysis~\cite{cookie_information2019cookie}.

\subsubsection{Our Work}
In our work, we address the limitations of previous auditing projects.
We provide a tool that offers an in-depth analysis, can be run on a wide variety of websites and is mostly automatic.
%For the analysis of cookie notices we will use NLP models developed by Bouhoula et al.~\cite{bouhoula2023automated} and the classification of cookies will be done with models developed by Bollinger et al.~\cite{bollinger2022automating}.
We use NLP models by Bouhoula et al~\cite{bouhoula2023automated} for the cookie notice analysis and gradient-boosting models by Bollinger et al.~\cite{bollinger2022automating} for the cookie classification.
Using them, we develop a browser extension that, requiring only minimal user involvement, interacts with a specific web page to audit its cookie compliance with the GDPR.

\subsubsection{Thesis Outline}